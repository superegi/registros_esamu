%\title{Reporte ingreso no autorizado sistema eSAMU}
\documentclass{article}
\usepackage[spanish, es-tabla]{babel}
\usepackage[utf8]{inputenc}

%\usepackage[utf8x]{inputenc}
\usepackage[T1]{fontenc}


%% Sets page size and margins
\usepackage[letterpaper,top=3.5cm,bottom=2.5cm,left=3cm,right=3cm,marginparwidth=1.75cm, headsep = 70pt]{geometry}

%% Useful packages
\usepackage{amsmath}
\usepackage{float}
\usepackage{graphicx}
\usepackage{multirow}
\usepackage{booktabs, makecell}
\usepackage[table,xcdraw]{xcolor}
\usepackage[colorinlistoftodos]{todonotes}
\usepackage[colorlinks=true, allcolors=blue]{hyperref}
\usepackage[hang, small,up,textfont=it,up]{caption} 
\usepackage{fancyhdr}
\usepackage{authblk}
\usepackage{csquotes}    %citas

\renewenvironment{quote}
  {\small\list{}{\rightmargin=3.5cm \leftmargin=3.5cm}%
   \item\relax}
  {\endlist}
  

%cabecera y pies
\pagestyle{fancy}
\fancyhf{}
\rhead{\includegraphics[width=0.12\textwidth]{SAMU_e.png}}
\lhead{
Reporte ingreso no autorizado sistema eSAMU \\ 
Dr E. Céspedes, Tecnologías SAMU \\
SAMU V Región \\
Agosto 2018}
\rfoot{P\'agina \thepage}

% titulo documento
\title{Indicadores de registro en eSAMU}
\author[1]{Dr E Céspedes}
\affil[1]{Tecnologías SAMU Viña del Mar}
\date{\today}


\begin{document}
\maketitle


\tableofcontents

\section{Introducción}
El sistema informático utilizado actualmente por el servicio de atención médica de urgencia (SAMU), fue un desarrollo del departamento de informática del Servicio de Salud Viña del Mar Quillota (SSVQ). El sistema es una plataforma diseñada para ser multipropósito, donde están incluídos varios módulos, uno de ellos es específicamente el que presta servios a SAMU en su día a día, llamado eSAMU. eSAMU se encuentra en la web  \url{http://mi.ssvq.cl}, disponible desde cualquier sito con conexión a internet, pero con clave de usuario única y personal.


Los indicadores
\begin{enumerate}

\item Registro inicial del Radiooperador
	\begin{enumerate}
	\item \emph{Atingencia motivo y submotivo llamada}: Si existe correlación entre el Motivo y submotivo de la llamada y el texto escrito. Se busca encontrar crrelación y concordancia. Indicador se mide en porcentaje.
	
	\item \emph{Registro del número telefónico del llamante}: Si existe registro del número de contacto con el solicitante

	\end{enumerate}
\item Regulador

\end{enumerate}
Registro inicial Radioopera
asfva



\section{Resultados}

\input{./hola1.txt}


\begin{table}[h]
\begin{tabular}{@{}lcc@{}}
\toprule
\multicolumn{1}{c}{\textbf{Nombre}} & \multicolumn{1}{c}{\textbf{Cantidad}} & \multicolumn{1}{c}{\textbf{\%}} \\ \midrule
Andres Exequiel Fuentes Moreno      & 72                                    & 36,92                           \\
Roberto Andres Valdebenito Paredes  & 42                                    & 21,54                           \\
Cristian Hernan Meins Lazo          & 42                                    & 21,54                           \\
Tomas antonio Tapia Vicencio        & 34                                    & 17,44                           \\
Juan Miguel Espinoza Juarez         & 2                                     & 1,03                            \\
Daniela Fernanda Olivares Torres    & 2                                     & 1,03                            \\
Ivan Andres Barahona Morales        & 1                                     & 0,51                            \\ \bottomrule
\end{tabular}
\end{table}

\section{Conclusión}
El 97,4\% de los ingresos es realizado por  los 4 usuarios de funcionarios listados en la tabla de los resultados. Dentro de los usuarios de funcionarios identificados los que tienen 2 y menos accesos se puede tratar como cifra despreciables y sin importancia.

Contrastando la realidad de posible fuga de información desde parte del SAMU, hay que ser cuidadosos al interpretar la información aquí entregada.



\end{document}

